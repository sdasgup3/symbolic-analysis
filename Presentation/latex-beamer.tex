%\documentclass{beamer}
\documentclass[mathserif,10pt]{beamer}

\usepackage{beamerthemesplit}
\usepackage{graphics}
\usepackage{epsfig}
\usepackage{algorithm}
\usepackage{verbatim}
\usepackage{listings}
\usepackage{framed}
\usepackage{pstricks}
\usepackage{pst-node,pst-tree}
\usepackage{pst-rel-points}
\usepackage{flexiprogram}
\usepackage[UKenglish]{babel}
\usepackage{hyperref}
\usepackage{pst-coil}
\usepackage{color}
\usepackage{epsfig}
\usepackage{tikz}
%\usepackage{multirow}

\usefonttheme{serif}

\newcommand{\cmt}[1]{}
%\noindent

\setcounter{tocdepth}{1}
\lstset{language=[ANSI]C}
\lstset{% general command to set parameter(s)
  basicstyle=\footnotesize\tt, % print whole listing small
    identifierstyle=, % nothing happens
    commentstyle=\color{red}, % white comments
    showstringspaces=false, % no special string spaces
    lineskip=1pt,
    captionpos=b,
    frame=single,
    breaklines=true
      %\insertauthor[width={3cm},center,respectlinebreaks]
}


\setbeamercovered{transparent=50}

\lstset{classoffset=0,
  morekeywords={},keywordstyle=\color{black},
  classoffset=1,
  classoffset=0}% restore default

  \usetheme{CambridgeUS}
  \usecolortheme{dolphin}

  \title[Bug Finding in Pointer Analysis]{Bug Finding in Pointer Analysis}
  \author[]{{\textbf{Theodoros, Sandeep}} }
  \begin{document}

  \begin{frame}
  \titlepage
  \end{frame}
  \usebeamertemplate{mytheme}

  \AtBeginSection[]
{
  \begin{frame}<beamer>
    \frametitle{Outline}
  \tableofcontents[currentsection]
    \end{frame}
}

\defverbatim[colored]\lstI{
\begin{lstlisting}[language=C++,basicstyle=\ttfamily,keywordstyle=\color{red}]
  int main() {
    int x=1 , y=2;
    int*  p = (int *)malloc(sizeof(int));

    klee_make_symbolic(&x, sizeof(x), "x");
    //klee_make_symbolic(&y, sizeof(y), "y");

    if(0 != x*y) {
      p = (int *)malloc(4);
    } else {
      if(y == 0) {
        p = (int *)malloc(4);
      }
    }
    return *p;
  }

\end{lstlisting}
}

\section{Implementation}
\subsection{Symbolic Execution}
\frame
{
  \frametitle{\subsecname}
  \begin{itemize}
    \item Symbolic execution using klee  
    \item Migration from Klee to Zesti (a variant of klee) 
    \cmt{
      To mitigate the path explosion problem, ZESTI carefully chooses divergent
        paths via two mechanisms: (1) it only diverges close to sensitive
        instructions (memory accesses and divisions.), i.e instructions that
        might contain a bug, and (2) it chooses the divergence points in order
        of increasing distance from the sensitive instruction. The key idea
        behind this approach is to exercise sensitive instructions on slightly
        different paths, with the goal of triggering a bug if the respective
        instructions contain one.  ZESTI identifies sensitive instructions
        dynamically while running the concrete input.  
    } 
  \end{itemize} 
}

\subsection{Checker Logic}
\frame
{
  \frametitle{\subsecname}
  \begin{itemize}
    \item Instrumenting the code to add checks. 
    \item Incorporating the checker logic in zesti.
    \cmt{
    This prevents zesti from interpreting the checker code (as now the checker code will be a part of the symbolic execution engine). 
    With this the run time performance is expected to improve.
      To mitigate the path explosion problem, ZESTI carefully chooses divergent
        paths via two mechanisms: (1) it only diverges close to sensitive
        instructions (memory accesses and divisions.), i.e instructions that
        might contain a bug, and (2) it chooses the divergence points in order
        of increasing distance from the sensitive instruction. The key idea
        behind this approach is to exercise sensitive instructions on slightly
        different paths, with the goal of triggering a bug if the respective
        instructions contain one.  ZESTI identifies sensitive instructions
        dynamically while running the concrete input.  
    } 
  \end{itemize} 
}

\subsection{Implicit klee\_assumes}
\frame
{
  \frametitle{\subsecname}

}

\subsection{Which variable to make symbolic}
\frame
{
  \frametitle{\subsecname}

}

\subsection{Making Input variable symbolic}
\frame
{
  \frametitle{\subsecname}
  \lstI

}

\subsection{Testing}
\frame
{
  \frametitle{\subsecname}

}

\section{Questions?}
\subsection{Questions?}
\frame
{}

\end{document}
